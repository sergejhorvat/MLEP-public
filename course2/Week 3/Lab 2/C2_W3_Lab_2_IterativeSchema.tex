\documentclass[11pt]{article}

    \usepackage[breakable]{tcolorbox}
    \usepackage{parskip} % Stop auto-indenting (to mimic markdown behaviour)
    
    \usepackage{iftex}
    \ifPDFTeX
    	\usepackage[T1]{fontenc}
    	\usepackage{mathpazo}
    \else
    	\usepackage{fontspec}
    \fi

    % Basic figure setup, for now with no caption control since it's done
    % automatically by Pandoc (which extracts ![](path) syntax from Markdown).
    \usepackage{graphicx}
    % Maintain compatibility with old templates. Remove in nbconvert 6.0
    \let\Oldincludegraphics\includegraphics
    % Ensure that by default, figures have no caption (until we provide a
    % proper Figure object with a Caption API and a way to capture that
    % in the conversion process - todo).
    \usepackage{caption}
    \DeclareCaptionFormat{nocaption}{}
    \captionsetup{format=nocaption,aboveskip=0pt,belowskip=0pt}

    \usepackage{float}
    \floatplacement{figure}{H} % forces figures to be placed at the correct location
    \usepackage{xcolor} % Allow colors to be defined
    \usepackage{enumerate} % Needed for markdown enumerations to work
    \usepackage{geometry} % Used to adjust the document margins
    \usepackage{amsmath} % Equations
    \usepackage{amssymb} % Equations
    \usepackage{textcomp} % defines textquotesingle
    % Hack from http://tex.stackexchange.com/a/47451/13684:
    \AtBeginDocument{%
        \def\PYZsq{\textquotesingle}% Upright quotes in Pygmentized code
    }
    \usepackage{upquote} % Upright quotes for verbatim code
    \usepackage{eurosym} % defines \euro
    \usepackage[mathletters]{ucs} % Extended unicode (utf-8) support
    \usepackage{fancyvrb} % verbatim replacement that allows latex
    \usepackage{grffile} % extends the file name processing of package graphics 
                         % to support a larger range
    \makeatletter % fix for old versions of grffile with XeLaTeX
    \@ifpackagelater{grffile}{2019/11/01}
    {
      % Do nothing on new versions
    }
    {
      \def\Gread@@xetex#1{%
        \IfFileExists{"\Gin@base".bb}%
        {\Gread@eps{\Gin@base.bb}}%
        {\Gread@@xetex@aux#1}%
      }
    }
    \makeatother
    \usepackage[Export]{adjustbox} % Used to constrain images to a maximum size
    \adjustboxset{max size={0.9\linewidth}{0.9\paperheight}}

    % The hyperref package gives us a pdf with properly built
    % internal navigation ('pdf bookmarks' for the table of contents,
    % internal cross-reference links, web links for URLs, etc.)
    \usepackage{hyperref}
    % The default LaTeX title has an obnoxious amount of whitespace. By default,
    % titling removes some of it. It also provides customization options.
    \usepackage{titling}
    \usepackage{longtable} % longtable support required by pandoc >1.10
    \usepackage{booktabs}  % table support for pandoc > 1.12.2
    \usepackage[inline]{enumitem} % IRkernel/repr support (it uses the enumerate* environment)
    \usepackage[normalem]{ulem} % ulem is needed to support strikethroughs (\sout)
                                % normalem makes italics be italics, not underlines
    \usepackage{mathrsfs}
    

    
    % Colors for the hyperref package
    \definecolor{urlcolor}{rgb}{0,.145,.698}
    \definecolor{linkcolor}{rgb}{.71,0.21,0.01}
    \definecolor{citecolor}{rgb}{.12,.54,.11}

    % ANSI colors
    \definecolor{ansi-black}{HTML}{3E424D}
    \definecolor{ansi-black-intense}{HTML}{282C36}
    \definecolor{ansi-red}{HTML}{E75C58}
    \definecolor{ansi-red-intense}{HTML}{B22B31}
    \definecolor{ansi-green}{HTML}{00A250}
    \definecolor{ansi-green-intense}{HTML}{007427}
    \definecolor{ansi-yellow}{HTML}{DDB62B}
    \definecolor{ansi-yellow-intense}{HTML}{B27D12}
    \definecolor{ansi-blue}{HTML}{208FFB}
    \definecolor{ansi-blue-intense}{HTML}{0065CA}
    \definecolor{ansi-magenta}{HTML}{D160C4}
    \definecolor{ansi-magenta-intense}{HTML}{A03196}
    \definecolor{ansi-cyan}{HTML}{60C6C8}
    \definecolor{ansi-cyan-intense}{HTML}{258F8F}
    \definecolor{ansi-white}{HTML}{C5C1B4}
    \definecolor{ansi-white-intense}{HTML}{A1A6B2}
    \definecolor{ansi-default-inverse-fg}{HTML}{FFFFFF}
    \definecolor{ansi-default-inverse-bg}{HTML}{000000}

    % common color for the border for error outputs.
    \definecolor{outerrorbackground}{HTML}{FFDFDF}

    % commands and environments needed by pandoc snippets
    % extracted from the output of `pandoc -s`
    \providecommand{\tightlist}{%
      \setlength{\itemsep}{0pt}\setlength{\parskip}{0pt}}
    \DefineVerbatimEnvironment{Highlighting}{Verbatim}{commandchars=\\\{\}}
    % Add ',fontsize=\small' for more characters per line
    \newenvironment{Shaded}{}{}
    \newcommand{\KeywordTok}[1]{\textcolor[rgb]{0.00,0.44,0.13}{\textbf{{#1}}}}
    \newcommand{\DataTypeTok}[1]{\textcolor[rgb]{0.56,0.13,0.00}{{#1}}}
    \newcommand{\DecValTok}[1]{\textcolor[rgb]{0.25,0.63,0.44}{{#1}}}
    \newcommand{\BaseNTok}[1]{\textcolor[rgb]{0.25,0.63,0.44}{{#1}}}
    \newcommand{\FloatTok}[1]{\textcolor[rgb]{0.25,0.63,0.44}{{#1}}}
    \newcommand{\CharTok}[1]{\textcolor[rgb]{0.25,0.44,0.63}{{#1}}}
    \newcommand{\StringTok}[1]{\textcolor[rgb]{0.25,0.44,0.63}{{#1}}}
    \newcommand{\CommentTok}[1]{\textcolor[rgb]{0.38,0.63,0.69}{\textit{{#1}}}}
    \newcommand{\OtherTok}[1]{\textcolor[rgb]{0.00,0.44,0.13}{{#1}}}
    \newcommand{\AlertTok}[1]{\textcolor[rgb]{1.00,0.00,0.00}{\textbf{{#1}}}}
    \newcommand{\FunctionTok}[1]{\textcolor[rgb]{0.02,0.16,0.49}{{#1}}}
    \newcommand{\RegionMarkerTok}[1]{{#1}}
    \newcommand{\ErrorTok}[1]{\textcolor[rgb]{1.00,0.00,0.00}{\textbf{{#1}}}}
    \newcommand{\NormalTok}[1]{{#1}}
    
    % Additional commands for more recent versions of Pandoc
    \newcommand{\ConstantTok}[1]{\textcolor[rgb]{0.53,0.00,0.00}{{#1}}}
    \newcommand{\SpecialCharTok}[1]{\textcolor[rgb]{0.25,0.44,0.63}{{#1}}}
    \newcommand{\VerbatimStringTok}[1]{\textcolor[rgb]{0.25,0.44,0.63}{{#1}}}
    \newcommand{\SpecialStringTok}[1]{\textcolor[rgb]{0.73,0.40,0.53}{{#1}}}
    \newcommand{\ImportTok}[1]{{#1}}
    \newcommand{\DocumentationTok}[1]{\textcolor[rgb]{0.73,0.13,0.13}{\textit{{#1}}}}
    \newcommand{\AnnotationTok}[1]{\textcolor[rgb]{0.38,0.63,0.69}{\textbf{\textit{{#1}}}}}
    \newcommand{\CommentVarTok}[1]{\textcolor[rgb]{0.38,0.63,0.69}{\textbf{\textit{{#1}}}}}
    \newcommand{\VariableTok}[1]{\textcolor[rgb]{0.10,0.09,0.49}{{#1}}}
    \newcommand{\ControlFlowTok}[1]{\textcolor[rgb]{0.00,0.44,0.13}{\textbf{{#1}}}}
    \newcommand{\OperatorTok}[1]{\textcolor[rgb]{0.40,0.40,0.40}{{#1}}}
    \newcommand{\BuiltInTok}[1]{{#1}}
    \newcommand{\ExtensionTok}[1]{{#1}}
    \newcommand{\PreprocessorTok}[1]{\textcolor[rgb]{0.74,0.48,0.00}{{#1}}}
    \newcommand{\AttributeTok}[1]{\textcolor[rgb]{0.49,0.56,0.16}{{#1}}}
    \newcommand{\InformationTok}[1]{\textcolor[rgb]{0.38,0.63,0.69}{\textbf{\textit{{#1}}}}}
    \newcommand{\WarningTok}[1]{\textcolor[rgb]{0.38,0.63,0.69}{\textbf{\textit{{#1}}}}}
    
    
    % Define a nice break command that doesn't care if a line doesn't already
    % exist.
    \def\br{\hspace*{\fill} \\* }
    % Math Jax compatibility definitions
    \def\gt{>}
    \def\lt{<}
    \let\Oldtex\TeX
    \let\Oldlatex\LaTeX
    \renewcommand{\TeX}{\textrm{\Oldtex}}
    \renewcommand{\LaTeX}{\textrm{\Oldlatex}}
    % Document parameters
    % Document title
    \title{C2\_W3\_Lab\_2\_IterativeSchema}
    
    
    
    
    
% Pygments definitions
\makeatletter
\def\PY@reset{\let\PY@it=\relax \let\PY@bf=\relax%
    \let\PY@ul=\relax \let\PY@tc=\relax%
    \let\PY@bc=\relax \let\PY@ff=\relax}
\def\PY@tok#1{\csname PY@tok@#1\endcsname}
\def\PY@toks#1+{\ifx\relax#1\empty\else%
    \PY@tok{#1}\expandafter\PY@toks\fi}
\def\PY@do#1{\PY@bc{\PY@tc{\PY@ul{%
    \PY@it{\PY@bf{\PY@ff{#1}}}}}}}
\def\PY#1#2{\PY@reset\PY@toks#1+\relax+\PY@do{#2}}

\@namedef{PY@tok@w}{\def\PY@tc##1{\textcolor[rgb]{0.73,0.73,0.73}{##1}}}
\@namedef{PY@tok@c}{\let\PY@it=\textit\def\PY@tc##1{\textcolor[rgb]{0.25,0.50,0.50}{##1}}}
\@namedef{PY@tok@cp}{\def\PY@tc##1{\textcolor[rgb]{0.74,0.48,0.00}{##1}}}
\@namedef{PY@tok@k}{\let\PY@bf=\textbf\def\PY@tc##1{\textcolor[rgb]{0.00,0.50,0.00}{##1}}}
\@namedef{PY@tok@kp}{\def\PY@tc##1{\textcolor[rgb]{0.00,0.50,0.00}{##1}}}
\@namedef{PY@tok@kt}{\def\PY@tc##1{\textcolor[rgb]{0.69,0.00,0.25}{##1}}}
\@namedef{PY@tok@o}{\def\PY@tc##1{\textcolor[rgb]{0.40,0.40,0.40}{##1}}}
\@namedef{PY@tok@ow}{\let\PY@bf=\textbf\def\PY@tc##1{\textcolor[rgb]{0.67,0.13,1.00}{##1}}}
\@namedef{PY@tok@nb}{\def\PY@tc##1{\textcolor[rgb]{0.00,0.50,0.00}{##1}}}
\@namedef{PY@tok@nf}{\def\PY@tc##1{\textcolor[rgb]{0.00,0.00,1.00}{##1}}}
\@namedef{PY@tok@nc}{\let\PY@bf=\textbf\def\PY@tc##1{\textcolor[rgb]{0.00,0.00,1.00}{##1}}}
\@namedef{PY@tok@nn}{\let\PY@bf=\textbf\def\PY@tc##1{\textcolor[rgb]{0.00,0.00,1.00}{##1}}}
\@namedef{PY@tok@ne}{\let\PY@bf=\textbf\def\PY@tc##1{\textcolor[rgb]{0.82,0.25,0.23}{##1}}}
\@namedef{PY@tok@nv}{\def\PY@tc##1{\textcolor[rgb]{0.10,0.09,0.49}{##1}}}
\@namedef{PY@tok@no}{\def\PY@tc##1{\textcolor[rgb]{0.53,0.00,0.00}{##1}}}
\@namedef{PY@tok@nl}{\def\PY@tc##1{\textcolor[rgb]{0.63,0.63,0.00}{##1}}}
\@namedef{PY@tok@ni}{\let\PY@bf=\textbf\def\PY@tc##1{\textcolor[rgb]{0.60,0.60,0.60}{##1}}}
\@namedef{PY@tok@na}{\def\PY@tc##1{\textcolor[rgb]{0.49,0.56,0.16}{##1}}}
\@namedef{PY@tok@nt}{\let\PY@bf=\textbf\def\PY@tc##1{\textcolor[rgb]{0.00,0.50,0.00}{##1}}}
\@namedef{PY@tok@nd}{\def\PY@tc##1{\textcolor[rgb]{0.67,0.13,1.00}{##1}}}
\@namedef{PY@tok@s}{\def\PY@tc##1{\textcolor[rgb]{0.73,0.13,0.13}{##1}}}
\@namedef{PY@tok@sd}{\let\PY@it=\textit\def\PY@tc##1{\textcolor[rgb]{0.73,0.13,0.13}{##1}}}
\@namedef{PY@tok@si}{\let\PY@bf=\textbf\def\PY@tc##1{\textcolor[rgb]{0.73,0.40,0.53}{##1}}}
\@namedef{PY@tok@se}{\let\PY@bf=\textbf\def\PY@tc##1{\textcolor[rgb]{0.73,0.40,0.13}{##1}}}
\@namedef{PY@tok@sr}{\def\PY@tc##1{\textcolor[rgb]{0.73,0.40,0.53}{##1}}}
\@namedef{PY@tok@ss}{\def\PY@tc##1{\textcolor[rgb]{0.10,0.09,0.49}{##1}}}
\@namedef{PY@tok@sx}{\def\PY@tc##1{\textcolor[rgb]{0.00,0.50,0.00}{##1}}}
\@namedef{PY@tok@m}{\def\PY@tc##1{\textcolor[rgb]{0.40,0.40,0.40}{##1}}}
\@namedef{PY@tok@gh}{\let\PY@bf=\textbf\def\PY@tc##1{\textcolor[rgb]{0.00,0.00,0.50}{##1}}}
\@namedef{PY@tok@gu}{\let\PY@bf=\textbf\def\PY@tc##1{\textcolor[rgb]{0.50,0.00,0.50}{##1}}}
\@namedef{PY@tok@gd}{\def\PY@tc##1{\textcolor[rgb]{0.63,0.00,0.00}{##1}}}
\@namedef{PY@tok@gi}{\def\PY@tc##1{\textcolor[rgb]{0.00,0.63,0.00}{##1}}}
\@namedef{PY@tok@gr}{\def\PY@tc##1{\textcolor[rgb]{1.00,0.00,0.00}{##1}}}
\@namedef{PY@tok@ge}{\let\PY@it=\textit}
\@namedef{PY@tok@gs}{\let\PY@bf=\textbf}
\@namedef{PY@tok@gp}{\let\PY@bf=\textbf\def\PY@tc##1{\textcolor[rgb]{0.00,0.00,0.50}{##1}}}
\@namedef{PY@tok@go}{\def\PY@tc##1{\textcolor[rgb]{0.53,0.53,0.53}{##1}}}
\@namedef{PY@tok@gt}{\def\PY@tc##1{\textcolor[rgb]{0.00,0.27,0.87}{##1}}}
\@namedef{PY@tok@err}{\def\PY@bc##1{{\setlength{\fboxsep}{\string -\fboxrule}\fcolorbox[rgb]{1.00,0.00,0.00}{1,1,1}{\strut ##1}}}}
\@namedef{PY@tok@kc}{\let\PY@bf=\textbf\def\PY@tc##1{\textcolor[rgb]{0.00,0.50,0.00}{##1}}}
\@namedef{PY@tok@kd}{\let\PY@bf=\textbf\def\PY@tc##1{\textcolor[rgb]{0.00,0.50,0.00}{##1}}}
\@namedef{PY@tok@kn}{\let\PY@bf=\textbf\def\PY@tc##1{\textcolor[rgb]{0.00,0.50,0.00}{##1}}}
\@namedef{PY@tok@kr}{\let\PY@bf=\textbf\def\PY@tc##1{\textcolor[rgb]{0.00,0.50,0.00}{##1}}}
\@namedef{PY@tok@bp}{\def\PY@tc##1{\textcolor[rgb]{0.00,0.50,0.00}{##1}}}
\@namedef{PY@tok@fm}{\def\PY@tc##1{\textcolor[rgb]{0.00,0.00,1.00}{##1}}}
\@namedef{PY@tok@vc}{\def\PY@tc##1{\textcolor[rgb]{0.10,0.09,0.49}{##1}}}
\@namedef{PY@tok@vg}{\def\PY@tc##1{\textcolor[rgb]{0.10,0.09,0.49}{##1}}}
\@namedef{PY@tok@vi}{\def\PY@tc##1{\textcolor[rgb]{0.10,0.09,0.49}{##1}}}
\@namedef{PY@tok@vm}{\def\PY@tc##1{\textcolor[rgb]{0.10,0.09,0.49}{##1}}}
\@namedef{PY@tok@sa}{\def\PY@tc##1{\textcolor[rgb]{0.73,0.13,0.13}{##1}}}
\@namedef{PY@tok@sb}{\def\PY@tc##1{\textcolor[rgb]{0.73,0.13,0.13}{##1}}}
\@namedef{PY@tok@sc}{\def\PY@tc##1{\textcolor[rgb]{0.73,0.13,0.13}{##1}}}
\@namedef{PY@tok@dl}{\def\PY@tc##1{\textcolor[rgb]{0.73,0.13,0.13}{##1}}}
\@namedef{PY@tok@s2}{\def\PY@tc##1{\textcolor[rgb]{0.73,0.13,0.13}{##1}}}
\@namedef{PY@tok@sh}{\def\PY@tc##1{\textcolor[rgb]{0.73,0.13,0.13}{##1}}}
\@namedef{PY@tok@s1}{\def\PY@tc##1{\textcolor[rgb]{0.73,0.13,0.13}{##1}}}
\@namedef{PY@tok@mb}{\def\PY@tc##1{\textcolor[rgb]{0.40,0.40,0.40}{##1}}}
\@namedef{PY@tok@mf}{\def\PY@tc##1{\textcolor[rgb]{0.40,0.40,0.40}{##1}}}
\@namedef{PY@tok@mh}{\def\PY@tc##1{\textcolor[rgb]{0.40,0.40,0.40}{##1}}}
\@namedef{PY@tok@mi}{\def\PY@tc##1{\textcolor[rgb]{0.40,0.40,0.40}{##1}}}
\@namedef{PY@tok@il}{\def\PY@tc##1{\textcolor[rgb]{0.40,0.40,0.40}{##1}}}
\@namedef{PY@tok@mo}{\def\PY@tc##1{\textcolor[rgb]{0.40,0.40,0.40}{##1}}}
\@namedef{PY@tok@ch}{\let\PY@it=\textit\def\PY@tc##1{\textcolor[rgb]{0.25,0.50,0.50}{##1}}}
\@namedef{PY@tok@cm}{\let\PY@it=\textit\def\PY@tc##1{\textcolor[rgb]{0.25,0.50,0.50}{##1}}}
\@namedef{PY@tok@cpf}{\let\PY@it=\textit\def\PY@tc##1{\textcolor[rgb]{0.25,0.50,0.50}{##1}}}
\@namedef{PY@tok@c1}{\let\PY@it=\textit\def\PY@tc##1{\textcolor[rgb]{0.25,0.50,0.50}{##1}}}
\@namedef{PY@tok@cs}{\let\PY@it=\textit\def\PY@tc##1{\textcolor[rgb]{0.25,0.50,0.50}{##1}}}

\def\PYZbs{\char`\\}
\def\PYZus{\char`\_}
\def\PYZob{\char`\{}
\def\PYZcb{\char`\}}
\def\PYZca{\char`\^}
\def\PYZam{\char`\&}
\def\PYZlt{\char`\<}
\def\PYZgt{\char`\>}
\def\PYZsh{\char`\#}
\def\PYZpc{\char`\%}
\def\PYZdl{\char`\$}
\def\PYZhy{\char`\-}
\def\PYZsq{\char`\'}
\def\PYZdq{\char`\"}
\def\PYZti{\char`\~}
% for compatibility with earlier versions
\def\PYZat{@}
\def\PYZlb{[}
\def\PYZrb{]}
\makeatother


    % For linebreaks inside Verbatim environment from package fancyvrb. 
    \makeatletter
        \newbox\Wrappedcontinuationbox 
        \newbox\Wrappedvisiblespacebox 
        \newcommand*\Wrappedvisiblespace {\textcolor{red}{\textvisiblespace}} 
        \newcommand*\Wrappedcontinuationsymbol {\textcolor{red}{\llap{\tiny$\m@th\hookrightarrow$}}} 
        \newcommand*\Wrappedcontinuationindent {3ex } 
        \newcommand*\Wrappedafterbreak {\kern\Wrappedcontinuationindent\copy\Wrappedcontinuationbox} 
        % Take advantage of the already applied Pygments mark-up to insert 
        % potential linebreaks for TeX processing. 
        %        {, <, #, %, $, ' and ": go to next line. 
        %        _, }, ^, &, >, - and ~: stay at end of broken line. 
        % Use of \textquotesingle for straight quote. 
        \newcommand*\Wrappedbreaksatspecials {% 
            \def\PYGZus{\discretionary{\char`\_}{\Wrappedafterbreak}{\char`\_}}% 
            \def\PYGZob{\discretionary{}{\Wrappedafterbreak\char`\{}{\char`\{}}% 
            \def\PYGZcb{\discretionary{\char`\}}{\Wrappedafterbreak}{\char`\}}}% 
            \def\PYGZca{\discretionary{\char`\^}{\Wrappedafterbreak}{\char`\^}}% 
            \def\PYGZam{\discretionary{\char`\&}{\Wrappedafterbreak}{\char`\&}}% 
            \def\PYGZlt{\discretionary{}{\Wrappedafterbreak\char`\<}{\char`\<}}% 
            \def\PYGZgt{\discretionary{\char`\>}{\Wrappedafterbreak}{\char`\>}}% 
            \def\PYGZsh{\discretionary{}{\Wrappedafterbreak\char`\#}{\char`\#}}% 
            \def\PYGZpc{\discretionary{}{\Wrappedafterbreak\char`\%}{\char`\%}}% 
            \def\PYGZdl{\discretionary{}{\Wrappedafterbreak\char`\$}{\char`\$}}% 
            \def\PYGZhy{\discretionary{\char`\-}{\Wrappedafterbreak}{\char`\-}}% 
            \def\PYGZsq{\discretionary{}{\Wrappedafterbreak\textquotesingle}{\textquotesingle}}% 
            \def\PYGZdq{\discretionary{}{\Wrappedafterbreak\char`\"}{\char`\"}}% 
            \def\PYGZti{\discretionary{\char`\~}{\Wrappedafterbreak}{\char`\~}}% 
        } 
        % Some characters . , ; ? ! / are not pygmentized. 
        % This macro makes them "active" and they will insert potential linebreaks 
        \newcommand*\Wrappedbreaksatpunct {% 
            \lccode`\~`\.\lowercase{\def~}{\discretionary{\hbox{\char`\.}}{\Wrappedafterbreak}{\hbox{\char`\.}}}% 
            \lccode`\~`\,\lowercase{\def~}{\discretionary{\hbox{\char`\,}}{\Wrappedafterbreak}{\hbox{\char`\,}}}% 
            \lccode`\~`\;\lowercase{\def~}{\discretionary{\hbox{\char`\;}}{\Wrappedafterbreak}{\hbox{\char`\;}}}% 
            \lccode`\~`\:\lowercase{\def~}{\discretionary{\hbox{\char`\:}}{\Wrappedafterbreak}{\hbox{\char`\:}}}% 
            \lccode`\~`\?\lowercase{\def~}{\discretionary{\hbox{\char`\?}}{\Wrappedafterbreak}{\hbox{\char`\?}}}% 
            \lccode`\~`\!\lowercase{\def~}{\discretionary{\hbox{\char`\!}}{\Wrappedafterbreak}{\hbox{\char`\!}}}% 
            \lccode`\~`\/\lowercase{\def~}{\discretionary{\hbox{\char`\/}}{\Wrappedafterbreak}{\hbox{\char`\/}}}% 
            \catcode`\.\active
            \catcode`\,\active 
            \catcode`\;\active
            \catcode`\:\active
            \catcode`\?\active
            \catcode`\!\active
            \catcode`\/\active 
            \lccode`\~`\~ 	
        }
    \makeatother

    \let\OriginalVerbatim=\Verbatim
    \makeatletter
    \renewcommand{\Verbatim}[1][1]{%
        %\parskip\z@skip
        \sbox\Wrappedcontinuationbox {\Wrappedcontinuationsymbol}%
        \sbox\Wrappedvisiblespacebox {\FV@SetupFont\Wrappedvisiblespace}%
        \def\FancyVerbFormatLine ##1{\hsize\linewidth
            \vtop{\raggedright\hyphenpenalty\z@\exhyphenpenalty\z@
                \doublehyphendemerits\z@\finalhyphendemerits\z@
                \strut ##1\strut}%
        }%
        % If the linebreak is at a space, the latter will be displayed as visible
        % space at end of first line, and a continuation symbol starts next line.
        % Stretch/shrink are however usually zero for typewriter font.
        \def\FV@Space {%
            \nobreak\hskip\z@ plus\fontdimen3\font minus\fontdimen4\font
            \discretionary{\copy\Wrappedvisiblespacebox}{\Wrappedafterbreak}
            {\kern\fontdimen2\font}%
        }%
        
        % Allow breaks at special characters using \PYG... macros.
        \Wrappedbreaksatspecials
        % Breaks at punctuation characters . , ; ? ! and / need catcode=\active 	
        \OriginalVerbatim[#1,codes*=\Wrappedbreaksatpunct]%
    }
    \makeatother

    % Exact colors from NB
    \definecolor{incolor}{HTML}{303F9F}
    \definecolor{outcolor}{HTML}{D84315}
    \definecolor{cellborder}{HTML}{CFCFCF}
    \definecolor{cellbackground}{HTML}{F7F7F7}
    
    % prompt
    \makeatletter
    \newcommand{\boxspacing}{\kern\kvtcb@left@rule\kern\kvtcb@boxsep}
    \makeatother
    \newcommand{\prompt}[4]{
        {\ttfamily\llap{{\color{#2}[#3]:\hspace{3pt}#4}}\vspace{-\baselineskip}}
    }
    

    
    % Prevent overflowing lines due to hard-to-break entities
    \sloppy 
    % Setup hyperref package
    \hypersetup{
      breaklinks=true,  % so long urls are correctly broken across lines
      colorlinks=true,
      urlcolor=urlcolor,
      linkcolor=linkcolor,
      citecolor=citecolor,
      }
    % Slightly bigger margins than the latex defaults
    
    \geometry{verbose,tmargin=1in,bmargin=1in,lmargin=1in,rmargin=1in}
    
    

\begin{document}
    
    \maketitle
    
    

    
    \hypertarget{ungraded-lab-iterative-schema-with-tfx-and-ml-metadata}{%
\section{Ungraded Lab: Iterative Schema with TFX and ML
Metadata}\label{ungraded-lab-iterative-schema-with-tfx-and-ml-metadata}}

In this notebook, you will get to review how to update an inferred
schema and save the result to the metadata store used by TFX. As
mentioned before, the TFX components get information from this database
before running executions. Thus, if you will be curating a schema, you
will need to save this as an artifact in the metadata store. You will
get to see how that is done in the following exercise.

Afterwards, you will also practice accessing the TFX metadata store and
see how you can track the lineage of an artifact.

    \hypertarget{setup}{%
\subsection{Setup}\label{setup}}

    \hypertarget{imports}{%
\subsubsection{Imports}\label{imports}}

    \begin{tcolorbox}[breakable, size=fbox, boxrule=1pt, pad at break*=1mm,colback=cellbackground, colframe=cellborder]
\prompt{In}{incolor}{1}{\boxspacing}
\begin{Verbatim}[commandchars=\\\{\}]
\PY{k+kn}{import} \PY{n+nn}{tensorflow} \PY{k}{as} \PY{n+nn}{tf}
\PY{k+kn}{import} \PY{n+nn}{tensorflow\PYZus{}data\PYZus{}validation} \PY{k}{as} \PY{n+nn}{tfdv}

\PY{k+kn}{from} \PY{n+nn}{tfx}\PY{n+nn}{.}\PY{n+nn}{components} \PY{k+kn}{import} \PY{n}{CsvExampleGen}
\PY{k+kn}{from} \PY{n+nn}{tfx}\PY{n+nn}{.}\PY{n+nn}{components} \PY{k+kn}{import} \PY{n}{ExampleValidator}
\PY{k+kn}{from} \PY{n+nn}{tfx}\PY{n+nn}{.}\PY{n+nn}{components} \PY{k+kn}{import} \PY{n}{SchemaGen}
\PY{k+kn}{from} \PY{n+nn}{tfx}\PY{n+nn}{.}\PY{n+nn}{components} \PY{k+kn}{import} \PY{n}{StatisticsGen}
\PY{k+kn}{from} \PY{n+nn}{tfx}\PY{n+nn}{.}\PY{n+nn}{components} \PY{k+kn}{import} \PY{n}{ImporterNode}
\PY{k+kn}{from} \PY{n+nn}{tfx}\PY{n+nn}{.}\PY{n+nn}{types} \PY{k+kn}{import} \PY{n}{standard\PYZus{}artifacts}

\PY{k+kn}{from} \PY{n+nn}{tfx}\PY{n+nn}{.}\PY{n+nn}{orchestration}\PY{n+nn}{.}\PY{n+nn}{experimental}\PY{n+nn}{.}\PY{n+nn}{interactive}\PY{n+nn}{.}\PY{n+nn}{interactive\PYZus{}context} \PY{k+kn}{import} \PY{n}{InteractiveContext}
\PY{k+kn}{from} \PY{n+nn}{google}\PY{n+nn}{.}\PY{n+nn}{protobuf}\PY{n+nn}{.}\PY{n+nn}{json\PYZus{}format} \PY{k+kn}{import} \PY{n}{MessageToDict}
\PY{k+kn}{from} \PY{n+nn}{tensorflow\PYZus{}metadata}\PY{n+nn}{.}\PY{n+nn}{proto}\PY{n+nn}{.}\PY{n+nn}{v0} \PY{k+kn}{import} \PY{n}{schema\PYZus{}pb2}

\PY{k+kn}{import} \PY{n+nn}{os}
\PY{k+kn}{import} \PY{n+nn}{pprint}
\PY{n}{pp} \PY{o}{=} \PY{n}{pprint}\PY{o}{.}\PY{n}{PrettyPrinter}\PY{p}{(}\PY{p}{)}
\end{Verbatim}
\end{tcolorbox}

    \hypertarget{define-paths}{%
\subsubsection{Define paths}\label{define-paths}}

For familiarity, you will again be using the
\href{https://archive.ics.uci.edu/ml/datasets/Adult}{Census Income
dataset} from the previous weeks' ungraded labs. You will use the same
paths to your raw data and pipeline files as shown below.

    \begin{tcolorbox}[breakable, size=fbox, boxrule=1pt, pad at break*=1mm,colback=cellbackground, colframe=cellborder]
\prompt{In}{incolor}{2}{\boxspacing}
\begin{Verbatim}[commandchars=\\\{\}]
\PY{c+c1}{\PYZsh{} location of the pipeline metadata store}
\PY{n}{\PYZus{}pipeline\PYZus{}root} \PY{o}{=} \PY{l+s+s1}{\PYZsq{}}\PY{l+s+s1}{./pipeline/}\PY{l+s+s1}{\PYZsq{}}

\PY{c+c1}{\PYZsh{} directory of the raw data files}
\PY{n}{\PYZus{}data\PYZus{}root} \PY{o}{=} \PY{l+s+s1}{\PYZsq{}}\PY{l+s+s1}{./data/census\PYZus{}data}\PY{l+s+s1}{\PYZsq{}}

\PY{c+c1}{\PYZsh{} path to the raw training data}
\PY{n}{\PYZus{}data\PYZus{}filepath} \PY{o}{=} \PY{n}{os}\PY{o}{.}\PY{n}{path}\PY{o}{.}\PY{n}{join}\PY{p}{(}\PY{n}{\PYZus{}data\PYZus{}root}\PY{p}{,} \PY{l+s+s1}{\PYZsq{}}\PY{l+s+s1}{adult.data}\PY{l+s+s1}{\PYZsq{}}\PY{p}{)}
\end{Verbatim}
\end{tcolorbox}

    \hypertarget{data-pipeline}{%
\subsection{Data Pipeline}\label{data-pipeline}}

Each TFX component you use accepts and generates artifacts which are
instances of the different artifact types TFX has configured in the
metadata store. The properties of these instances are shown neatly in a
table in the outputs of \texttt{context.run()}. TFX does all of these
for you so you only need to inspect the output of each component to know
which property of the artifact you can pass on to the next component
(e.g.~the
\texttt{outputs{[}\textquotesingle{}examples\textquotesingle{}{]}} of
\texttt{ExampleGen} can be passed to \texttt{StatisticsGen}).

Since you've already used this dataset before, we will just quickly go
over \texttt{ExampleGen}, \texttt{StatisticsGen}, and
\texttt{SchemaGen}. The new concepts will be discussed after the said
components.

    \hypertarget{create-the-interactive-context}{%
\subsubsection{Create the Interactive
Context}\label{create-the-interactive-context}}

    \begin{tcolorbox}[breakable, size=fbox, boxrule=1pt, pad at break*=1mm,colback=cellbackground, colframe=cellborder]
\prompt{In}{incolor}{3}{\boxspacing}
\begin{Verbatim}[commandchars=\\\{\}]
\PY{c+c1}{\PYZsh{} Initialize the InteractiveContext.}
\PY{c+c1}{\PYZsh{} If you leave `\PYZus{}pipeline\PYZus{}root` blank, then the db will be created in a temporary directory.}
\PY{n}{context} \PY{o}{=} \PY{n}{InteractiveContext}\PY{p}{(}\PY{n}{pipeline\PYZus{}root}\PY{o}{=}\PY{n}{\PYZus{}pipeline\PYZus{}root}\PY{p}{)}
\end{Verbatim}
\end{tcolorbox}

    \begin{Verbatim}[commandchars=\\\{\}]
WARNING:absl:InteractiveContext metadata\_connection\_config not provided: using
SQLite ML Metadata database at ./pipeline/metadata.sqlite.
    \end{Verbatim}

    \hypertarget{examplegen}{%
\subsubsection{ExampleGen}\label{examplegen}}

    \begin{tcolorbox}[breakable, size=fbox, boxrule=1pt, pad at break*=1mm,colback=cellbackground, colframe=cellborder]
\prompt{In}{incolor}{4}{\boxspacing}
\begin{Verbatim}[commandchars=\\\{\}]
\PY{c+c1}{\PYZsh{} Instantiate ExampleGen with the input CSV dataset}
\PY{n}{example\PYZus{}gen} \PY{o}{=} \PY{n}{CsvExampleGen}\PY{p}{(}\PY{n}{input\PYZus{}base}\PY{o}{=}\PY{n}{\PYZus{}data\PYZus{}root}\PY{p}{)}

\PY{c+c1}{\PYZsh{} Execute the component}
\PY{n}{context}\PY{o}{.}\PY{n}{run}\PY{p}{(}\PY{n}{example\PYZus{}gen}\PY{p}{)}
\end{Verbatim}
\end{tcolorbox}

    
    
            \begin{tcolorbox}[breakable, size=fbox, boxrule=.5pt, pad at break*=1mm, opacityfill=0]
\prompt{Out}{outcolor}{4}{\boxspacing}
\begin{Verbatim}[commandchars=\\\{\}]
ExecutionResult(
    component\_id: CsvExampleGen
    execution\_id: 1
    outputs:
        examples: Channel(
            type\_name: Examples
            artifacts: [Artifact(artifact: id: 1
        type\_id: 5
        uri: "./pipeline/CsvExampleGen/examples/1"
        properties \{
          key: "split\_names"
          value \{
            string\_value: "[\textbackslash{}"train\textbackslash{}", \textbackslash{}"eval\textbackslash{}"]"
          \}
        \}
        custom\_properties \{
          key: "input\_fingerprint"
          value \{
            string\_value: "split:single\_split,num\_files:1,total\_bytes:3974460,xo
r\_checksum:1618242085,sum\_checksum:1618242085"
          \}
        \}
        custom\_properties \{
          key: "payload\_format"
          value \{
            string\_value: "FORMAT\_TF\_EXAMPLE"
          \}
        \}
        custom\_properties \{
          key: "span"
          value \{
            string\_value: "0"
          \}
        \}
        custom\_properties \{
          key: "state"
          value \{
            string\_value: "published"
          \}
        \}
        , artifact\_type: id: 5
        name: "Examples"
        properties \{
          key: "span"
          value: INT
        \}
        properties \{
          key: "split\_names"
          value: STRING
        \}
        properties \{
          key: "version"
          value: INT
        \}
        )]
        ))
\end{Verbatim}
\end{tcolorbox}
        
    \hypertarget{statisticsgen}{%
\subsubsection{StatisticsGen}\label{statisticsgen}}

    \begin{tcolorbox}[breakable, size=fbox, boxrule=1pt, pad at break*=1mm,colback=cellbackground, colframe=cellborder]
\prompt{In}{incolor}{5}{\boxspacing}
\begin{Verbatim}[commandchars=\\\{\}]
\PY{c+c1}{\PYZsh{} Instantiate StatisticsGen with the ExampleGen ingested dataset}
\PY{n}{statistics\PYZus{}gen} \PY{o}{=} \PY{n}{StatisticsGen}\PY{p}{(}
    \PY{n}{examples}\PY{o}{=}\PY{n}{example\PYZus{}gen}\PY{o}{.}\PY{n}{outputs}\PY{p}{[}\PY{l+s+s1}{\PYZsq{}}\PY{l+s+s1}{examples}\PY{l+s+s1}{\PYZsq{}}\PY{p}{]}\PY{p}{)}

\PY{c+c1}{\PYZsh{} Execute the component}
\PY{n}{context}\PY{o}{.}\PY{n}{run}\PY{p}{(}\PY{n}{statistics\PYZus{}gen}\PY{p}{)}
\end{Verbatim}
\end{tcolorbox}

            \begin{tcolorbox}[breakable, size=fbox, boxrule=.5pt, pad at break*=1mm, opacityfill=0]
\prompt{Out}{outcolor}{5}{\boxspacing}
\begin{Verbatim}[commandchars=\\\{\}]
ExecutionResult(
    component\_id: StatisticsGen
    execution\_id: 2
    outputs:
        statistics: Channel(
            type\_name: ExampleStatistics
            artifacts: [Artifact(artifact: id: 2
        type\_id: 7
        uri: "./pipeline/StatisticsGen/statistics/2"
        properties \{
          key: "split\_names"
          value \{
            string\_value: "[\textbackslash{}"train\textbackslash{}", \textbackslash{}"eval\textbackslash{}"]"
          \}
        \}
        custom\_properties \{
          key: "name"
          value \{
            string\_value: "statistics"
          \}
        \}
        custom\_properties \{
          key: "producer\_component"
          value \{
            string\_value: "StatisticsGen"
          \}
        \}
        custom\_properties \{
          key: "state"
          value \{
            string\_value: "published"
          \}
        \}
        , artifact\_type: id: 7
        name: "ExampleStatistics"
        properties \{
          key: "span"
          value: INT
        \}
        properties \{
          key: "split\_names"
          value: STRING
        \}
        )]
        ))
\end{Verbatim}
\end{tcolorbox}
        
    \hypertarget{schemagen}{%
\subsubsection{SchemaGen}\label{schemagen}}

    \begin{tcolorbox}[breakable, size=fbox, boxrule=1pt, pad at break*=1mm,colback=cellbackground, colframe=cellborder]
\prompt{In}{incolor}{6}{\boxspacing}
\begin{Verbatim}[commandchars=\\\{\}]
\PY{c+c1}{\PYZsh{} Instantiate SchemaGen with the StatisticsGen ingested dataset}
\PY{n}{schema\PYZus{}gen} \PY{o}{=} \PY{n}{SchemaGen}\PY{p}{(}
    \PY{n}{statistics}\PY{o}{=}\PY{n}{statistics\PYZus{}gen}\PY{o}{.}\PY{n}{outputs}\PY{p}{[}\PY{l+s+s1}{\PYZsq{}}\PY{l+s+s1}{statistics}\PY{l+s+s1}{\PYZsq{}}\PY{p}{]}\PY{p}{,}
    \PY{p}{)}

\PY{c+c1}{\PYZsh{} Run the component}
\PY{n}{context}\PY{o}{.}\PY{n}{run}\PY{p}{(}\PY{n}{schema\PYZus{}gen}\PY{p}{)}
\end{Verbatim}
\end{tcolorbox}

    \begin{Verbatim}[commandchars=\\\{\}]
WARNING:tensorflow:From /opt/conda/lib/python3.8/site-
packages/tensorflow\_data\_validation/utils/stats\_util.py:229: tf\_record\_iterator
(from tensorflow.python.lib.io.tf\_record) is deprecated and will be removed in a
future version.
Instructions for updating:
Use eager execution and:
`tf.data.TFRecordDataset(path)`
    \end{Verbatim}

            \begin{tcolorbox}[breakable, size=fbox, boxrule=.5pt, pad at break*=1mm, opacityfill=0]
\prompt{Out}{outcolor}{6}{\boxspacing}
\begin{Verbatim}[commandchars=\\\{\}]
ExecutionResult(
    component\_id: SchemaGen
    execution\_id: 3
    outputs:
        schema: Channel(
            type\_name: Schema
            artifacts: [Artifact(artifact: id: 3
        type\_id: 9
        uri: "./pipeline/SchemaGen/schema/3"
        custom\_properties \{
          key: "name"
          value \{
            string\_value: "schema"
          \}
        \}
        custom\_properties \{
          key: "producer\_component"
          value \{
            string\_value: "SchemaGen"
          \}
        \}
        custom\_properties \{
          key: "state"
          value \{
            string\_value: "published"
          \}
        \}
        , artifact\_type: id: 9
        name: "Schema"
        )]
        ))
\end{Verbatim}
\end{tcolorbox}
        
    \begin{tcolorbox}[breakable, size=fbox, boxrule=1pt, pad at break*=1mm,colback=cellbackground, colframe=cellborder]
\prompt{In}{incolor}{7}{\boxspacing}
\begin{Verbatim}[commandchars=\\\{\}]
\PY{c+c1}{\PYZsh{} Visualize the schema}
\PY{n}{context}\PY{o}{.}\PY{n}{show}\PY{p}{(}\PY{n}{schema\PYZus{}gen}\PY{o}{.}\PY{n}{outputs}\PY{p}{[}\PY{l+s+s1}{\PYZsq{}}\PY{l+s+s1}{schema}\PY{l+s+s1}{\PYZsq{}}\PY{p}{]}\PY{p}{)}
\end{Verbatim}
\end{tcolorbox}

    
    \begin{Verbatim}[commandchars=\\\{\}]
<IPython.core.display.HTML object>
    \end{Verbatim}

    
    
    \begin{Verbatim}[commandchars=\\\{\}]
                    Type  Presence Valency            Domain
Feature name                                                
'age'                INT  required  single                 -
'capital-gain'       INT  required  single                 -
'capital-loss'       INT  required  single                 -
'education'       STRING  required  single       'education'
'education-num'      INT  required  single                 -
'fnlwgt'             INT  required  single                 -
'hours-per-week'     INT  required  single                 -
'label'           STRING  required  single           'label'
'marital-status'  STRING  required  single  'marital-status'
'native-country'  STRING  required  single  'native-country'
'occupation'      STRING  required  single      'occupation'
'race'            STRING  required  single            'race'
'relationship'    STRING  required  single    'relationship'
'sex'             STRING  required  single             'sex'
'workclass'       STRING  required  single       'workclass'
    \end{Verbatim}

    
    
    \begin{Verbatim}[commandchars=\\\{\}]
                                                                                                                                                                                                                                                                                                                                                                                                                                                                                                                                                                           Values
Domain                                                                                                                                                                                                                                                                                                                                                                                                                                                                                                                                                                           
'education'       ' 10th', ' 11th', ' 12th', ' 1st-4th', ' 5th-6th', ' 7th-8th', ' 9th', ' Assoc-acdm', ' Assoc-voc', ' Bachelors', ' Doctorate', ' HS-grad', ' Masters', ' Preschool', ' Prof-school', ' Some-college'                                                                                                                                                                                                                                                                                                                                                          
'label'           ' <=50K', ' >50K'                                                                                                                                                                                                                                                                                                                                                                                                                                                                                                                                              
'marital-status'  ' Divorced', ' Married-AF-spouse', ' Married-civ-spouse', ' Married-spouse-absent', ' Never-married', ' Separated', ' Widowed'                                                                                                                                                                                                                                                                                                                                                                                                                                 
'native-country'  ' ?', ' Cambodia', ' Canada', ' China', ' Columbia', ' Cuba', ' Dominican-Republic', ' Ecuador', ' El-Salvador', ' England', ' France', ' Germany', ' Greece', ' Guatemala', ' Haiti', ' Honduras', ' Hong', ' Hungary', ' India', ' Iran', ' Ireland', ' Italy', ' Jamaica', ' Japan', ' Laos', ' Mexico', ' Nicaragua', ' Outlying-US(Guam-USVI-etc)', ' Peru', ' Philippines', ' Poland', ' Portugal', ' Puerto-Rico', ' Scotland', ' South', ' Taiwan', ' Thailand', ' Trinadad\&Tobago', ' United-States', ' Vietnam', ' Yugoslavia', ' Holand-Netherlands'
'occupation'      ' ?', ' Adm-clerical', ' Armed-Forces', ' Craft-repair', ' Exec-managerial', ' Farming-fishing', ' Handlers-cleaners', ' Machine-op-inspct', ' Other-service', ' Priv-house-serv', ' Prof-specialty', ' Protective-serv', ' Sales', ' Tech-support', ' Transport-moving'                                                                                                                                                                                                                                                                                       
'race'            ' Amer-Indian-Eskimo', ' Asian-Pac-Islander', ' Black', ' Other', ' White'                                                                                                                                                                                                                                                                                                                                                                                                                                                                                     
'relationship'    ' Husband', ' Not-in-family', ' Other-relative', ' Own-child', ' Unmarried', ' Wife'                                                                                                                                                                                                                                                                                                                                                                                                                                                                           
'sex'             ' Female', ' Male'                                                                                                                                                                                                                                                                                                                                                                                                                                                                                                                                             
'workclass'       ' ?', ' Federal-gov', ' Local-gov', ' Never-worked', ' Private', ' Self-emp-inc', ' Self-emp-not-inc', ' State-gov', ' Without-pay'                                                                                                                                                                                                                                                                                                                                                                                                                            
    \end{Verbatim}

    
    \hypertarget{curating-the-schema}{%
\subsubsection{Curating the Schema}\label{curating-the-schema}}

    Now that you have the inferred schema, you can proceed to revising it to
be more robust. For instance, you can restrict the age as you did in
Week 1. First, you have to load the \texttt{Schema} protocol buffer from
the metadata store. You can do this by getting the schema uri from the
output of \texttt{SchemaGen} then use TFDV's
\texttt{load\_schema\_text()} method.

    \begin{tcolorbox}[breakable, size=fbox, boxrule=1pt, pad at break*=1mm,colback=cellbackground, colframe=cellborder]
\prompt{In}{incolor}{8}{\boxspacing}
\begin{Verbatim}[commandchars=\\\{\}]
\PY{c+c1}{\PYZsh{} Get the schema uri}
\PY{n}{schema\PYZus{}uri} \PY{o}{=} \PY{n}{schema\PYZus{}gen}\PY{o}{.}\PY{n}{outputs}\PY{p}{[}\PY{l+s+s1}{\PYZsq{}}\PY{l+s+s1}{schema}\PY{l+s+s1}{\PYZsq{}}\PY{p}{]}\PY{o}{.}\PY{n}{\PYZus{}artifacts}\PY{p}{[}\PY{l+m+mi}{0}\PY{p}{]}\PY{o}{.}\PY{n}{uri}

\PY{c+c1}{\PYZsh{} Get the schema pbtxt file from the SchemaGen output}
\PY{n}{schema} \PY{o}{=} \PY{n}{tfdv}\PY{o}{.}\PY{n}{load\PYZus{}schema\PYZus{}text}\PY{p}{(}\PY{n}{os}\PY{o}{.}\PY{n}{path}\PY{o}{.}\PY{n}{join}\PY{p}{(}\PY{n}{schema\PYZus{}uri}\PY{p}{,} \PY{l+s+s1}{\PYZsq{}}\PY{l+s+s1}{schema.pbtxt}\PY{l+s+s1}{\PYZsq{}}\PY{p}{)}\PY{p}{)}
\end{Verbatim}
\end{tcolorbox}

    With that, you can now make changes to the schema as before. For the
purpose of this exercise, you will only modify the age domain but feel
free to add more if you want.

    \begin{tcolorbox}[breakable, size=fbox, boxrule=1pt, pad at break*=1mm,colback=cellbackground, colframe=cellborder]
\prompt{In}{incolor}{9}{\boxspacing}
\begin{Verbatim}[commandchars=\\\{\}]
\PY{c+c1}{\PYZsh{} Restrict the range of the `age` feature}
\PY{n}{tfdv}\PY{o}{.}\PY{n}{set\PYZus{}domain}\PY{p}{(}\PY{n}{schema}\PY{p}{,} \PY{l+s+s1}{\PYZsq{}}\PY{l+s+s1}{age}\PY{l+s+s1}{\PYZsq{}}\PY{p}{,} \PY{n}{schema\PYZus{}pb2}\PY{o}{.}\PY{n}{IntDomain}\PY{p}{(}\PY{n}{name}\PY{o}{=}\PY{l+s+s1}{\PYZsq{}}\PY{l+s+s1}{age}\PY{l+s+s1}{\PYZsq{}}\PY{p}{,} \PY{n+nb}{min}\PY{o}{=}\PY{l+m+mi}{17}\PY{p}{,} \PY{n+nb}{max}\PY{o}{=}\PY{l+m+mi}{90}\PY{p}{)}\PY{p}{)}

\PY{c+c1}{\PYZsh{} Display the modified schema. Notice the `Domain` column of `age`.}
\PY{n}{tfdv}\PY{o}{.}\PY{n}{display\PYZus{}schema}\PY{p}{(}\PY{n}{schema}\PY{p}{)}
\end{Verbatim}
\end{tcolorbox}

    
    \begin{Verbatim}[commandchars=\\\{\}]
                    Type  Presence Valency            Domain
Feature name                                                
'age'             INT     required  single  [17,90]         
'capital-gain'    INT     required  single  -               
'capital-loss'    INT     required  single  -               
'education'       STRING  required  single  'education'     
'education-num'   INT     required  single  -               
'fnlwgt'          INT     required  single  -               
'hours-per-week'  INT     required  single  -               
'label'           STRING  required  single  'label'         
'marital-status'  STRING  required  single  'marital-status'
'native-country'  STRING  required  single  'native-country'
'occupation'      STRING  required  single  'occupation'    
'race'            STRING  required  single  'race'          
'relationship'    STRING  required  single  'relationship'  
'sex'             STRING  required  single  'sex'           
'workclass'       STRING  required  single  'workclass'     
    \end{Verbatim}

    
    
    \begin{Verbatim}[commandchars=\\\{\}]
                                                                                                                                                                                                                                                                                                                                                                                                                                                                                                                                                                           Values
Domain                                                                                                                                                                                                                                                                                                                                                                                                                                                                                                                                                                           
'education'       ' 10th', ' 11th', ' 12th', ' 1st-4th', ' 5th-6th', ' 7th-8th', ' 9th', ' Assoc-acdm', ' Assoc-voc', ' Bachelors', ' Doctorate', ' HS-grad', ' Masters', ' Preschool', ' Prof-school', ' Some-college'                                                                                                                                                                                                                                                                                                                                                          
'label'           ' <=50K', ' >50K'                                                                                                                                                                                                                                                                                                                                                                                                                                                                                                                                              
'marital-status'  ' Divorced', ' Married-AF-spouse', ' Married-civ-spouse', ' Married-spouse-absent', ' Never-married', ' Separated', ' Widowed'                                                                                                                                                                                                                                                                                                                                                                                                                                 
'native-country'  ' ?', ' Cambodia', ' Canada', ' China', ' Columbia', ' Cuba', ' Dominican-Republic', ' Ecuador', ' El-Salvador', ' England', ' France', ' Germany', ' Greece', ' Guatemala', ' Haiti', ' Honduras', ' Hong', ' Hungary', ' India', ' Iran', ' Ireland', ' Italy', ' Jamaica', ' Japan', ' Laos', ' Mexico', ' Nicaragua', ' Outlying-US(Guam-USVI-etc)', ' Peru', ' Philippines', ' Poland', ' Portugal', ' Puerto-Rico', ' Scotland', ' South', ' Taiwan', ' Thailand', ' Trinadad\&Tobago', ' United-States', ' Vietnam', ' Yugoslavia', ' Holand-Netherlands'
'occupation'      ' ?', ' Adm-clerical', ' Armed-Forces', ' Craft-repair', ' Exec-managerial', ' Farming-fishing', ' Handlers-cleaners', ' Machine-op-inspct', ' Other-service', ' Priv-house-serv', ' Prof-specialty', ' Protective-serv', ' Sales', ' Tech-support', ' Transport-moving'                                                                                                                                                                                                                                                                                       
'race'            ' Amer-Indian-Eskimo', ' Asian-Pac-Islander', ' Black', ' Other', ' White'                                                                                                                                                                                                                                                                                                                                                                                                                                                                                     
'relationship'    ' Husband', ' Not-in-family', ' Other-relative', ' Own-child', ' Unmarried', ' Wife'                                                                                                                                                                                                                                                                                                                                                                                                                                                                           
'sex'             ' Female', ' Male'                                                                                                                                                                                                                                                                                                                                                                                                                                                                                                                                             
'workclass'       ' ?', ' Federal-gov', ' Local-gov', ' Never-worked', ' Private', ' Self-emp-inc', ' Self-emp-not-inc', ' State-gov', ' Without-pay'                                                                                                                                                                                                                                                                                                                                                                                                                            
    \end{Verbatim}

    
    \hypertarget{schema-environments}{%
\subsubsection{Schema Environments}\label{schema-environments}}

By default, your schema will watch for all the features declared above
including the label. However, when the model is served for inference, it
will get datasets that will not have the label because that is the
feature that the model will be trying to predict. You need to configure
the pipeline to not raise an alarm when this kind of dataset is
received.

You can do that with
\href{https://www.tensorflow.org/tfx/tutorials/data_validation/tfdv_basic\#schema_environments}{schema
environments}. First, you will need to declare training and serving
environments, then configure the serving schema to not watch for the
presence of labels. See how it is implemented below.

    \begin{tcolorbox}[breakable, size=fbox, boxrule=1pt, pad at break*=1mm,colback=cellbackground, colframe=cellborder]
\prompt{In}{incolor}{10}{\boxspacing}
\begin{Verbatim}[commandchars=\\\{\}]
\PY{c+c1}{\PYZsh{} Create schema environments for training and serving}
\PY{n}{schema}\PY{o}{.}\PY{n}{default\PYZus{}environment}\PY{o}{.}\PY{n}{append}\PY{p}{(}\PY{l+s+s1}{\PYZsq{}}\PY{l+s+s1}{TRAINING}\PY{l+s+s1}{\PYZsq{}}\PY{p}{)}
\PY{n}{schema}\PY{o}{.}\PY{n}{default\PYZus{}environment}\PY{o}{.}\PY{n}{append}\PY{p}{(}\PY{l+s+s1}{\PYZsq{}}\PY{l+s+s1}{SERVING}\PY{l+s+s1}{\PYZsq{}}\PY{p}{)}

\PY{c+c1}{\PYZsh{} Omit label from the serving environment}
\PY{n}{tfdv}\PY{o}{.}\PY{n}{get\PYZus{}feature}\PY{p}{(}\PY{n}{schema}\PY{p}{,} \PY{l+s+s1}{\PYZsq{}}\PY{l+s+s1}{label}\PY{l+s+s1}{\PYZsq{}}\PY{p}{)}\PY{o}{.}\PY{n}{not\PYZus{}in\PYZus{}environment}\PY{o}{.}\PY{n}{append}\PY{p}{(}\PY{l+s+s1}{\PYZsq{}}\PY{l+s+s1}{SERVING}\PY{l+s+s1}{\PYZsq{}}\PY{p}{)}
\end{Verbatim}
\end{tcolorbox}

    You can now freeze the curated schema and save to a local directory.

    \begin{tcolorbox}[breakable, size=fbox, boxrule=1pt, pad at break*=1mm,colback=cellbackground, colframe=cellborder]
\prompt{In}{incolor}{11}{\boxspacing}
\begin{Verbatim}[commandchars=\\\{\}]
\PY{c+c1}{\PYZsh{} Declare the path to the updated schema directory}
\PY{n}{\PYZus{}updated\PYZus{}schema\PYZus{}dir} \PY{o}{=} \PY{l+s+sa}{f}\PY{l+s+s1}{\PYZsq{}}\PY{l+s+si}{\PYZob{}}\PY{n}{\PYZus{}pipeline\PYZus{}root}\PY{l+s+si}{\PYZcb{}}\PY{l+s+s1}{/updated\PYZus{}schema}\PY{l+s+s1}{\PYZsq{}}

\PY{c+c1}{\PYZsh{} Create the said directory}
\PY{o}{!}mkdir \PYZhy{}p \PY{o}{\PYZob{}}\PYZus{}updated\PYZus{}schema\PYZus{}dir\PY{o}{\PYZcb{}}

\PY{c+c1}{\PYZsh{} Declare the path to the schema file}
\PY{n}{schema\PYZus{}file} \PY{o}{=} \PY{n}{os}\PY{o}{.}\PY{n}{path}\PY{o}{.}\PY{n}{join}\PY{p}{(}\PY{n}{\PYZus{}updated\PYZus{}schema\PYZus{}dir}\PY{p}{,} \PY{l+s+s1}{\PYZsq{}}\PY{l+s+s1}{schema.pbtxt}\PY{l+s+s1}{\PYZsq{}}\PY{p}{)}

\PY{c+c1}{\PYZsh{} Save the curated schema to the said file}
\PY{n}{tfdv}\PY{o}{.}\PY{n}{write\PYZus{}schema\PYZus{}text}\PY{p}{(}\PY{n}{schema}\PY{p}{,} \PY{n}{schema\PYZus{}file}\PY{p}{)}
\end{Verbatim}
\end{tcolorbox}

    \hypertarget{importernode}{%
\subsubsection{ImporterNode}\label{importernode}}

Now that the schema has been saved, you need to create an artifact in
the metadata store that will point to it. TFX provides the
\href{https://www.tensorflow.org/tfx/guide/statsgen\#using_the_statsgen_component_with_a_schema}{ImporterNode}
component used to import external objects to ML Metadata. You will need
to pass in the URI of the object and what type of artifact it is. See
the syntax below.

    \begin{tcolorbox}[breakable, size=fbox, boxrule=1pt, pad at break*=1mm,colback=cellbackground, colframe=cellborder]
\prompt{In}{incolor}{12}{\boxspacing}
\begin{Verbatim}[commandchars=\\\{\}]
\PY{c+c1}{\PYZsh{} Use an ImporterNode to put the curated schema to ML Metadata}
\PY{n}{user\PYZus{}schema\PYZus{}importer} \PY{o}{=} \PY{n}{ImporterNode}\PY{p}{(}
    \PY{n}{instance\PYZus{}name}\PY{o}{=}\PY{l+s+s1}{\PYZsq{}}\PY{l+s+s1}{import\PYZus{}user\PYZus{}schema}\PY{l+s+s1}{\PYZsq{}}\PY{p}{,}
    \PY{n}{source\PYZus{}uri}\PY{o}{=}\PY{n}{\PYZus{}updated\PYZus{}schema\PYZus{}dir}\PY{p}{,}
    \PY{n}{artifact\PYZus{}type}\PY{o}{=}\PY{n}{standard\PYZus{}artifacts}\PY{o}{.}\PY{n}{Schema}
\PY{p}{)}

\PY{c+c1}{\PYZsh{} Run the component}
\PY{n}{context}\PY{o}{.}\PY{n}{run}\PY{p}{(}\PY{n}{user\PYZus{}schema\PYZus{}importer}\PY{p}{,} \PY{n}{enable\PYZus{}cache}\PY{o}{=}\PY{k+kc}{False}\PY{p}{)}
\end{Verbatim}
\end{tcolorbox}

            \begin{tcolorbox}[breakable, size=fbox, boxrule=.5pt, pad at break*=1mm, opacityfill=0]
\prompt{Out}{outcolor}{12}{\boxspacing}
\begin{Verbatim}[commandchars=\\\{\}]
ExecutionResult(
    component\_id: ImporterNode.import\_user\_schema
    execution\_id: 4
    outputs:
        result: Channel(
            type\_name: Schema
            artifacts: [Artifact(artifact: id: 4
        type\_id: 9
        uri: "./pipeline//updated\_schema"
        , artifact\_type: id: 9
        name: "Schema"
        )]
        ))
\end{Verbatim}
\end{tcolorbox}
        
    If you pass in the component output to \texttt{context.show()}, then you
should see the schema.

    \begin{tcolorbox}[breakable, size=fbox, boxrule=1pt, pad at break*=1mm,colback=cellbackground, colframe=cellborder]
\prompt{In}{incolor}{13}{\boxspacing}
\begin{Verbatim}[commandchars=\\\{\}]
\PY{c+c1}{\PYZsh{} See the result}
\PY{n}{context}\PY{o}{.}\PY{n}{show}\PY{p}{(}\PY{n}{user\PYZus{}schema\PYZus{}importer}\PY{o}{.}\PY{n}{outputs}\PY{p}{[}\PY{l+s+s1}{\PYZsq{}}\PY{l+s+s1}{result}\PY{l+s+s1}{\PYZsq{}}\PY{p}{]}\PY{p}{)}
\end{Verbatim}
\end{tcolorbox}

    
    \begin{Verbatim}[commandchars=\\\{\}]
<IPython.core.display.HTML object>
    \end{Verbatim}

    
    
    \begin{Verbatim}[commandchars=\\\{\}]
                    Type  Presence Valency            Domain
Feature name                                                
'age'             INT     required  single  [17,90]         
'capital-gain'    INT     required  single  -               
'capital-loss'    INT     required  single  -               
'education'       STRING  required  single  'education'     
'education-num'   INT     required  single  -               
'fnlwgt'          INT     required  single  -               
'hours-per-week'  INT     required  single  -               
'label'           STRING  required  single  'label'         
'marital-status'  STRING  required  single  'marital-status'
'native-country'  STRING  required  single  'native-country'
'occupation'      STRING  required  single  'occupation'    
'race'            STRING  required  single  'race'          
'relationship'    STRING  required  single  'relationship'  
'sex'             STRING  required  single  'sex'           
'workclass'       STRING  required  single  'workclass'     
    \end{Verbatim}

    
    
    \begin{Verbatim}[commandchars=\\\{\}]
                                                                                                                                                                                                                                                                                                                                                                                                                                                                                                                                                                           Values
Domain                                                                                                                                                                                                                                                                                                                                                                                                                                                                                                                                                                           
'education'       ' 10th', ' 11th', ' 12th', ' 1st-4th', ' 5th-6th', ' 7th-8th', ' 9th', ' Assoc-acdm', ' Assoc-voc', ' Bachelors', ' Doctorate', ' HS-grad', ' Masters', ' Preschool', ' Prof-school', ' Some-college'                                                                                                                                                                                                                                                                                                                                                          
'label'           ' <=50K', ' >50K'                                                                                                                                                                                                                                                                                                                                                                                                                                                                                                                                              
'marital-status'  ' Divorced', ' Married-AF-spouse', ' Married-civ-spouse', ' Married-spouse-absent', ' Never-married', ' Separated', ' Widowed'                                                                                                                                                                                                                                                                                                                                                                                                                                 
'native-country'  ' ?', ' Cambodia', ' Canada', ' China', ' Columbia', ' Cuba', ' Dominican-Republic', ' Ecuador', ' El-Salvador', ' England', ' France', ' Germany', ' Greece', ' Guatemala', ' Haiti', ' Honduras', ' Hong', ' Hungary', ' India', ' Iran', ' Ireland', ' Italy', ' Jamaica', ' Japan', ' Laos', ' Mexico', ' Nicaragua', ' Outlying-US(Guam-USVI-etc)', ' Peru', ' Philippines', ' Poland', ' Portugal', ' Puerto-Rico', ' Scotland', ' South', ' Taiwan', ' Thailand', ' Trinadad\&Tobago', ' United-States', ' Vietnam', ' Yugoslavia', ' Holand-Netherlands'
'occupation'      ' ?', ' Adm-clerical', ' Armed-Forces', ' Craft-repair', ' Exec-managerial', ' Farming-fishing', ' Handlers-cleaners', ' Machine-op-inspct', ' Other-service', ' Priv-house-serv', ' Prof-specialty', ' Protective-serv', ' Sales', ' Tech-support', ' Transport-moving'                                                                                                                                                                                                                                                                                       
'race'            ' Amer-Indian-Eskimo', ' Asian-Pac-Islander', ' Black', ' Other', ' White'                                                                                                                                                                                                                                                                                                                                                                                                                                                                                     
'relationship'    ' Husband', ' Not-in-family', ' Other-relative', ' Own-child', ' Unmarried', ' Wife'                                                                                                                                                                                                                                                                                                                                                                                                                                                                           
'sex'             ' Female', ' Male'                                                                                                                                                                                                                                                                                                                                                                                                                                                                                                                                             
'workclass'       ' ?', ' Federal-gov', ' Local-gov', ' Never-worked', ' Private', ' Self-emp-inc', ' Self-emp-not-inc', ' State-gov', ' Without-pay'                                                                                                                                                                                                                                                                                                                                                                                                                            
    \end{Verbatim}

    
    \hypertarget{examplevalidator}{%
\subsubsection{ExampleValidator}\label{examplevalidator}}

You can then use this new artifact as input to the other components of
the pipeline. See how it is used as the \texttt{schema} argument in
\texttt{ExampleValidator} below.

    \begin{tcolorbox}[breakable, size=fbox, boxrule=1pt, pad at break*=1mm,colback=cellbackground, colframe=cellborder]
\prompt{In}{incolor}{14}{\boxspacing}
\begin{Verbatim}[commandchars=\\\{\}]
\PY{c+c1}{\PYZsh{} Instantiate ExampleValidator with the StatisticsGen and SchemaGen ingested data}
\PY{n}{example\PYZus{}validator} \PY{o}{=} \PY{n}{ExampleValidator}\PY{p}{(}
    \PY{n}{statistics}\PY{o}{=}\PY{n}{statistics\PYZus{}gen}\PY{o}{.}\PY{n}{outputs}\PY{p}{[}\PY{l+s+s1}{\PYZsq{}}\PY{l+s+s1}{statistics}\PY{l+s+s1}{\PYZsq{}}\PY{p}{]}\PY{p}{,}
    \PY{n}{schema}\PY{o}{=}\PY{n}{user\PYZus{}schema\PYZus{}importer}\PY{o}{.}\PY{n}{outputs}\PY{p}{[}\PY{l+s+s1}{\PYZsq{}}\PY{l+s+s1}{result}\PY{l+s+s1}{\PYZsq{}}\PY{p}{]}\PY{p}{)}

\PY{c+c1}{\PYZsh{} Run the component.}
\PY{n}{context}\PY{o}{.}\PY{n}{run}\PY{p}{(}\PY{n}{example\PYZus{}validator}\PY{p}{)}
\end{Verbatim}
\end{tcolorbox}

            \begin{tcolorbox}[breakable, size=fbox, boxrule=.5pt, pad at break*=1mm, opacityfill=0]
\prompt{Out}{outcolor}{14}{\boxspacing}
\begin{Verbatim}[commandchars=\\\{\}]
ExecutionResult(
    component\_id: ExampleValidator
    execution\_id: 5
    outputs:
        anomalies: Channel(
            type\_name: ExampleAnomalies
            artifacts: [Artifact(artifact: id: 5
        type\_id: 12
        uri: "./pipeline/ExampleValidator/anomalies/5"
        properties \{
          key: "split\_names"
          value \{
            string\_value: "[\textbackslash{}"train\textbackslash{}", \textbackslash{}"eval\textbackslash{}"]"
          \}
        \}
        custom\_properties \{
          key: "name"
          value \{
            string\_value: "anomalies"
          \}
        \}
        custom\_properties \{
          key: "producer\_component"
          value \{
            string\_value: "ExampleValidator"
          \}
        \}
        custom\_properties \{
          key: "state"
          value \{
            string\_value: "published"
          \}
        \}
        , artifact\_type: id: 12
        name: "ExampleAnomalies"
        properties \{
          key: "span"
          value: INT
        \}
        properties \{
          key: "split\_names"
          value: STRING
        \}
        )]
        ))
\end{Verbatim}
\end{tcolorbox}
        
    \begin{tcolorbox}[breakable, size=fbox, boxrule=1pt, pad at break*=1mm,colback=cellbackground, colframe=cellborder]
\prompt{In}{incolor}{15}{\boxspacing}
\begin{Verbatim}[commandchars=\\\{\}]
\PY{c+c1}{\PYZsh{} Visualize the results}
\PY{n}{context}\PY{o}{.}\PY{n}{show}\PY{p}{(}\PY{n}{example\PYZus{}validator}\PY{o}{.}\PY{n}{outputs}\PY{p}{[}\PY{l+s+s1}{\PYZsq{}}\PY{l+s+s1}{anomalies}\PY{l+s+s1}{\PYZsq{}}\PY{p}{]}\PY{p}{)}
\end{Verbatim}
\end{tcolorbox}

    
    \begin{Verbatim}[commandchars=\\\{\}]
<IPython.core.display.HTML object>
    \end{Verbatim}

    
    
    \begin{Verbatim}[commandchars=\\\{\}]
<IPython.core.display.HTML object>
    \end{Verbatim}

    
    
    \begin{Verbatim}[commandchars=\\\{\}]
<IPython.core.display.HTML object>
    \end{Verbatim}

    
    
    \begin{Verbatim}[commandchars=\\\{\}]
<IPython.core.display.HTML object>
    \end{Verbatim}

    
    
    \begin{Verbatim}[commandchars=\\\{\}]
<IPython.core.display.HTML object>
    \end{Verbatim}

    
    \hypertarget{practice-with-ml-metadata}{%
\subsubsection{Practice with ML
Metadata}\label{practice-with-ml-metadata}}

At this point, you should now take some time exploring the contents of
the metadata store saved by your component runs. This will let you
practice tracking artifacts and how they are related to each other. This
involves looking at artifacts, executions, and events. This skill will
let you recover related artifacts even without seeing the code of the
training run. All you need is access to the metadata store.

See how the input artifact IDs to an instance of
\texttt{ExampleAnomalies} are tracked in the following cells. If you
have this notebook, then you will already know that it uses the output
of StatisticsGen for this run and also the curated schema you imported.
However, if you already have hundreds of training runs and parameter
iterations, you may find it hard to track which is which. That's where
the metadata store will be useful. Since it records information about a
specific pipeline run, you will be able to track the inputs and outputs
of a particular artifact.

You will start by setting the connection config to the metadata store.

    \begin{tcolorbox}[breakable, size=fbox, boxrule=1pt, pad at break*=1mm,colback=cellbackground, colframe=cellborder]
\prompt{In}{incolor}{16}{\boxspacing}
\begin{Verbatim}[commandchars=\\\{\}]
\PY{c+c1}{\PYZsh{} Import mlmd and utilities}
\PY{k+kn}{import} \PY{n+nn}{ml\PYZus{}metadata} \PY{k}{as} \PY{n+nn}{mlmd}
\PY{k+kn}{from} \PY{n+nn}{ml\PYZus{}metadata}\PY{n+nn}{.}\PY{n+nn}{proto} \PY{k+kn}{import} \PY{n}{metadata\PYZus{}store\PYZus{}pb2}

\PY{c+c1}{\PYZsh{} Get the connection config to connect to the context\PYZsq{}s metadata store}
\PY{n}{connection\PYZus{}config} \PY{o}{=} \PY{n}{context}\PY{o}{.}\PY{n}{metadata\PYZus{}connection\PYZus{}config}

\PY{c+c1}{\PYZsh{} Instantiate a MetadataStore instance with the connection config}
\PY{n}{store} \PY{o}{=} \PY{n}{mlmd}\PY{o}{.}\PY{n}{MetadataStore}\PY{p}{(}\PY{n}{connection\PYZus{}config}\PY{p}{)}
\end{Verbatim}
\end{tcolorbox}

    Next, let's see what artifact types are available in the metadata store.

    \begin{tcolorbox}[breakable, size=fbox, boxrule=1pt, pad at break*=1mm,colback=cellbackground, colframe=cellborder]
\prompt{In}{incolor}{17}{\boxspacing}
\begin{Verbatim}[commandchars=\\\{\}]
\PY{c+c1}{\PYZsh{} Get artifact types}
\PY{n}{artifact\PYZus{}types} \PY{o}{=} \PY{n}{store}\PY{o}{.}\PY{n}{get\PYZus{}artifact\PYZus{}types}\PY{p}{(}\PY{p}{)}

\PY{c+c1}{\PYZsh{} Print the results}
\PY{p}{[}\PY{n}{artifact\PYZus{}type}\PY{o}{.}\PY{n}{name} \PY{k}{for} \PY{n}{artifact\PYZus{}type} \PY{o+ow}{in} \PY{n}{artifact\PYZus{}types}\PY{p}{]}
\end{Verbatim}
\end{tcolorbox}

            \begin{tcolorbox}[breakable, size=fbox, boxrule=.5pt, pad at break*=1mm, opacityfill=0]
\prompt{Out}{outcolor}{17}{\boxspacing}
\begin{Verbatim}[commandchars=\\\{\}]
['Examples', 'ExampleStatistics', 'Schema', 'ExampleAnomalies']
\end{Verbatim}
\end{tcolorbox}
        
    If you get the artifacts of type \texttt{Schema}, you will see that
there are two entries. One is the inferred and the other is the one you
imported. At the end of this exercise, you can verify that the curated
schema is the one used for the \texttt{ExampleValidator} run we will be
investigating.

    \begin{tcolorbox}[breakable, size=fbox, boxrule=1pt, pad at break*=1mm,colback=cellbackground, colframe=cellborder]
\prompt{In}{incolor}{18}{\boxspacing}
\begin{Verbatim}[commandchars=\\\{\}]
\PY{c+c1}{\PYZsh{} Get artifact types}
\PY{n}{schema\PYZus{}list} \PY{o}{=} \PY{n}{store}\PY{o}{.}\PY{n}{get\PYZus{}artifacts\PYZus{}by\PYZus{}type}\PY{p}{(}\PY{l+s+s1}{\PYZsq{}}\PY{l+s+s1}{Schema}\PY{l+s+s1}{\PYZsq{}}\PY{p}{)}

\PY{p}{[}\PY{p}{(}\PY{l+s+sa}{f}\PY{l+s+s1}{\PYZsq{}}\PY{l+s+s1}{schema uri: }\PY{l+s+si}{\PYZob{}}\PY{n}{schema}\PY{o}{.}\PY{n}{uri}\PY{l+s+si}{\PYZcb{}}\PY{l+s+s1}{\PYZsq{}}\PY{p}{,} \PY{l+s+sa}{f}\PY{l+s+s1}{\PYZsq{}}\PY{l+s+s1}{schema id:}\PY{l+s+si}{\PYZob{}}\PY{n}{schema}\PY{o}{.}\PY{n}{id}\PY{l+s+si}{\PYZcb{}}\PY{l+s+s1}{\PYZsq{}}\PY{p}{)} \PY{k}{for} \PY{n}{schema} \PY{o+ow}{in} \PY{n}{schema\PYZus{}list}\PY{p}{]}
\end{Verbatim}
\end{tcolorbox}

            \begin{tcolorbox}[breakable, size=fbox, boxrule=.5pt, pad at break*=1mm, opacityfill=0]
\prompt{Out}{outcolor}{18}{\boxspacing}
\begin{Verbatim}[commandchars=\\\{\}]
[('schema uri: ./pipeline/SchemaGen/schema/3', 'schema id:3'),
 ('schema uri: ./pipeline//updated\_schema', 'schema id:4')]
\end{Verbatim}
\end{tcolorbox}
        
    Let's get the first instance of \texttt{ExampleAnomalies} to get the
output of \texttt{ExampleValidator}.

    \begin{tcolorbox}[breakable, size=fbox, boxrule=1pt, pad at break*=1mm,colback=cellbackground, colframe=cellborder]
\prompt{In}{incolor}{19}{\boxspacing}
\begin{Verbatim}[commandchars=\\\{\}]
\PY{c+c1}{\PYZsh{} Get 1st instance of ExampleAnomalies}
\PY{n}{example\PYZus{}anomalies} \PY{o}{=} \PY{n}{store}\PY{o}{.}\PY{n}{get\PYZus{}artifacts\PYZus{}by\PYZus{}type}\PY{p}{(}\PY{l+s+s1}{\PYZsq{}}\PY{l+s+s1}{ExampleAnomalies}\PY{l+s+s1}{\PYZsq{}}\PY{p}{)}\PY{p}{[}\PY{l+m+mi}{0}\PY{p}{]}

\PY{c+c1}{\PYZsh{} Print the artifact id}
\PY{n+nb}{print}\PY{p}{(}\PY{l+s+sa}{f}\PY{l+s+s1}{\PYZsq{}}\PY{l+s+s1}{Artifact id: }\PY{l+s+si}{\PYZob{}}\PY{n}{example\PYZus{}anomalies}\PY{o}{.}\PY{n}{id}\PY{l+s+si}{\PYZcb{}}\PY{l+s+s1}{\PYZsq{}}\PY{p}{)}
\end{Verbatim}
\end{tcolorbox}

    \begin{Verbatim}[commandchars=\\\{\}]
Artifact id: 5
    \end{Verbatim}

    You will use the artifact ID to get events related to it. Let's just get
the first instance.

    \begin{tcolorbox}[breakable, size=fbox, boxrule=1pt, pad at break*=1mm,colback=cellbackground, colframe=cellborder]
\prompt{In}{incolor}{20}{\boxspacing}
\begin{Verbatim}[commandchars=\\\{\}]
\PY{c+c1}{\PYZsh{} Get first event related to the ID}
\PY{n}{anomalies\PYZus{}id\PYZus{}event} \PY{o}{=} \PY{n}{store}\PY{o}{.}\PY{n}{get\PYZus{}events\PYZus{}by\PYZus{}artifact\PYZus{}ids}\PY{p}{(}\PY{p}{[}\PY{n}{example\PYZus{}anomalies}\PY{o}{.}\PY{n}{id}\PY{p}{]}\PY{p}{)}\PY{p}{[}\PY{l+m+mi}{0}\PY{p}{]}

\PY{c+c1}{\PYZsh{} Print results}
\PY{n+nb}{print}\PY{p}{(}\PY{n}{anomalies\PYZus{}id\PYZus{}event}\PY{p}{)}
\end{Verbatim}
\end{tcolorbox}

    \begin{Verbatim}[commandchars=\\\{\}]
artifact\_id: 5
execution\_id: 5
path \{
  steps \{
    key: "anomalies"
  \}
  steps \{
    index: 0
  \}
\}
type: OUTPUT
milliseconds\_since\_epoch: 1625830637413

    \end{Verbatim}

    As expected, the event type will be an \texttt{OUTPUT} because this is
the output of the \texttt{ExampleValidator} component. Since we want to
get the inputs, we can track it through the execution id.

    \begin{tcolorbox}[breakable, size=fbox, boxrule=1pt, pad at break*=1mm,colback=cellbackground, colframe=cellborder]
\prompt{In}{incolor}{21}{\boxspacing}
\begin{Verbatim}[commandchars=\\\{\}]
\PY{c+c1}{\PYZsh{} Get execution ID}
\PY{n}{anomalies\PYZus{}execution\PYZus{}id} \PY{o}{=} \PY{n}{anomalies\PYZus{}id\PYZus{}event}\PY{o}{.}\PY{n}{execution\PYZus{}id}

\PY{c+c1}{\PYZsh{} Get events by the execution ID}
\PY{n}{events\PYZus{}execution} \PY{o}{=} \PY{n}{store}\PY{o}{.}\PY{n}{get\PYZus{}events\PYZus{}by\PYZus{}execution\PYZus{}ids}\PY{p}{(}\PY{p}{[}\PY{n}{anomalies\PYZus{}execution\PYZus{}id}\PY{p}{]}\PY{p}{)}

\PY{c+c1}{\PYZsh{} Print results}
\PY{n+nb}{print}\PY{p}{(}\PY{n}{events\PYZus{}execution}\PY{p}{)}
\end{Verbatim}
\end{tcolorbox}

    \begin{Verbatim}[commandchars=\\\{\}]
[artifact\_id: 2
execution\_id: 5
path \{
  steps \{
    key: "statistics"
  \}
  steps \{
    index: 0
  \}
\}
type: INPUT
milliseconds\_since\_epoch: 1625830637021
, artifact\_id: 4
execution\_id: 5
path \{
  steps \{
    key: "schema"
  \}
  steps \{
    index: 0
  \}
\}
type: INPUT
milliseconds\_since\_epoch: 1625830637021
, artifact\_id: 5
execution\_id: 5
path \{
  steps \{
    key: "anomalies"
  \}
  steps \{
    index: 0
  \}
\}
type: OUTPUT
milliseconds\_since\_epoch: 1625830637413
]
    \end{Verbatim}

    We see the artifacts which are marked as \texttt{INPUT} above
representing the statistics and schema inputs. We can extract their IDs
programmatically like this. You will see that you will get the artifact
ID of the curated schema you printed out earlier.

    \begin{tcolorbox}[breakable, size=fbox, boxrule=1pt, pad at break*=1mm,colback=cellbackground, colframe=cellborder]
\prompt{In}{incolor}{22}{\boxspacing}
\begin{Verbatim}[commandchars=\\\{\}]
\PY{c+c1}{\PYZsh{} Filter INPUT type events}
\PY{n}{inputs\PYZus{}to\PYZus{}exval} \PY{o}{=} \PY{p}{[}\PY{n}{event}\PY{o}{.}\PY{n}{artifact\PYZus{}id} \PY{k}{for} \PY{n}{event} \PY{o+ow}{in} \PY{n}{events\PYZus{}execution} 
                       \PY{k}{if} \PY{n}{event}\PY{o}{.}\PY{n}{type} \PY{o}{==} \PY{n}{metadata\PYZus{}store\PYZus{}pb2}\PY{o}{.}\PY{n}{Event}\PY{o}{.}\PY{n}{INPUT}\PY{p}{]}

\PY{c+c1}{\PYZsh{} Print results}
\PY{n+nb}{print}\PY{p}{(}\PY{n}{inputs\PYZus{}to\PYZus{}exval}\PY{p}{)}
\end{Verbatim}
\end{tcolorbox}

    \begin{Verbatim}[commandchars=\\\{\}]
[2, 4]
    \end{Verbatim}

    \textbf{Congratulations!} You have now completed this notebook on
iterative schemas and saw how it can be used in a TFX pipeline. You were
also able to track an artifact's lineage by looking at the artifacts,
events, and executions in the metadata store. These will come in handy
in this week's assignment!


    % Add a bibliography block to the postdoc
    
    
    
\end{document}
